\documentclass[11pt]{homework}

\usepackage{graphicx}
\usepackage[utf8]{inputenc}

\newcommand{\hwname}{Tomáš Sasák}
\newcommand{\hwemail}{xsasak01}
\newcommand{\hwtype}{Protokol}
\newcommand{\hwnum}{}
\newcommand{\hwclass}{}
\newcommand{\hwlecture}{Signály a systémy}
\newcommand{\hwsection}{}

\begin{document}
\maketitle

\question*{Úloha}

	Signál som načítal pomocou \verb|wavfile| a vynormoval. Počet vzorkov vypočítal podelením počtu vzorkov a vzorkovaciej frekvencie. Počet symbolov podelením počtu vzorkov a čísla 16.	

	Výsledky:
	\newline
	Vzorkovacia frekvencia: 16000 $[Hz]$
	\newline
	Počet vzorkov signálu: 32000 $[Vzorok]$
	\newline	
	Dĺžka signálu: 2.0 $[s]$
	\newline 
	Počet binárnych symbolov signálu: 2000 $[binarny symbol]$
	


\question*{Úloha}
Signál $s[n]$, prešiel \verb|for| cyklom. kde pre každú skupinu 16 prvkov, každý 8. prvok bol porovnaný a podľa zadaných podmienok bolo pridané dané binárne číslo 
$(0/1)$ do premennej \verb|symbols|. \newline
Pomocou nástroja \verb|diff|, som porovnal, zhodu vypočítaných dekódovaných binárnych symbolov so zadanými binárnymi symbolmi, ktoré sa zhodovali. Následne, bolo vypočítané, koľko vzorkov náleží $20ms$ a kolko času pripadá na jeden vzorok. \newline
Zadaný signál $s[n]$, bol vykreslený pre časový interval $0.020s$. Následne dekódované binárne symboly boli vykreslené ďalším \verb|for| cyklom  ktorý pre každý 16 prvok signálu $s[n]$, vygeneroval tento symbol pomocou \verb|stem|.\newline
Vznikol následujúci graf:
\begin{center}
\includegraphics[scale=0.4]{uloha2.pdf}
\end{center}

\question*{Úloha}
Nuly a póly filtra som vypočítal pomocou \verb|tf2zpk|. Pre stabilitu filtra musí platiť, že jeho póly sa musia nachádzať v jednotkovej kružnici. \newline
Vznikol následujúci graf:
\begin{center}
\includegraphics[scale=0.4]{uloha3.pdf}
\end{center}
Filter je stabilný.

\question*{Úloha}
Frekvenčnú charakteristiku som vypočítal pomocou \verb|freqz|. Pre výpočet modula, bolo treba odstrániť imaginárnu časť charakteristiky.\newline
Vznikol následujúci graf:
\begin{center}
\includegraphics[scale=0.4]{uloha4.pdf}
\end{center}
Tento filter je dolná priepusť.
Mezná frekvencia je lokálne maximum v intervale $<0,1000>$ na X osy, presnejšie 484.375 $[Hz]$

\question*{Úloha}
Signál som prefiltroval pomocou \verb|lfilter|.\newline
Vznikol následujúci graf:
\begin{center}
\includegraphics[scale=0.4]{uloha5.pdf}
\end{center}
Graf som si v prieberu riešenia presnejšie vykreslil a uznal som z neho, že signál je posunutý približne 0.001 $[s]$ doprava, čo udáva 16 vzorkov.\newline
\question*{Úloha}
Signál som posunul. Symboly posunutého signálu som zdekódoval rovnakým \verb|for| cyklom ako v $2.$ úlohe.\newline
Vznikol následujúci graf:
\begin{center}
\includegraphics[scale=0.35]{uloha6.pdf}
\end{center}
\question*{Úloha}
Symboly patriace zadanému signálu som skrátil o 1 prvok od konca a porovnal pomocou \verb|logical_xor|.
Výsledky:\newline
Počet chybných bitov vzniknutých posunom: 95 $[vzorok]$ \newline
Symboly dekódované z ss\_shifted majú chybovosť: 4.7524 \% \newline
\question*{Úloha}
Spektra signálov som vypočítal pomocou \verb|fft|.
\begin{center}
\includegraphics[scale=0.4]{uloha8.pdf}
\end{center}
Spektra sú si trocha podobné len zozačiatku, ale neskôr už je vidieť že signál bol prefiltrovaným zadaným filtrom. Toto rovnako podporuje tvrdenie, že filter je dolná priepusť a že od istej frekvencie prestane prepúštať vzorky signálu.\newline
\question*{Úloha}
Funkciu hustoty pravdepodobnosti som vypočítal pomocou \verb|histogram| a podelil prvky histogramu počtom realizácií signálu $s[n]$ a veľkosťou jedného binu histogramu.
\begin{center}
\includegraphics[scale=0.4]{uloha9.pdf}
\end{center}
Kontrola integrálu $p(x)$: 1.0000 \newline
\question*{Úloha}
Koreláciu signálu som vykonal pomocou \verb|corerlate| a vykreslil som výsledky len od $<-50,50>$\newline
Vznikol následujúci graf:
\begin{center}
\includegraphics[scale=0.4]{uloha10.pdf}
\end{center}
\question*{Úloha}
$R[0]$, je presný stred grafu, čiže index presne v strede. $R[1]$, stred + 1 index. $R[2]$, stred + 2 index.
Výsledky: \newline

Hodnota koeficientu $R[0]$: 0.23300420415043482 \newline
Hodnota koeficientu $R[1]:$ 0.26962989249519886 \newline
Hodnota koeficientu $R[16]$: 0.01985382855273201 \newline
\question*{Úloha}
Časový odhad združenej funkcie hustoty rozdelenia pravdepodobnosti som vypočítal pomocou \verb|histogram2d| a \verb|meshgrid|. \newline
Vznikol následujúci graf:
\begin{center}
\includegraphics[scale=0.4]{uloha12.pdf}
\end{center}
\question*{Úloha}
Pre výpočet integrálu bolo nutné vypočítať veľkosť jedného binu, a výnasobiť ju hodnotami grafu $p(x_1, x_2, 1)$ a sčítať.
\begin{verbatim}
binsize = (numpy.abs(x1_edges[0] - x1_edges[1])) * (numpy.abs(x2_edges[0] - x2_edges[1]))
numpy.sum(px1x2 * binsize)
\end{verbatim}
Kontrola integrálu  $p(x_1, x_2, 1)$: 0.9999999999999934\newline
\question*{Úloha}
Pre výpočet integrálu bolo nutné, osekať $X$ a $Y$  pre $R[0]$ a následne vynásobiť tieto $X$, $Y$ a hodnoty grafu $p(x_1, x_2, 1)$ a spraviť sumu. Následovne bolo treba vynásobiť ešte s veľkosťou jedného binu a spraviť sumu.
\begin{verbatim}
r0 = numpy.sum(numpy.sum(X_correct * Y_correct * px1x2) * binsize)
\end{verbatim}
$R[0]$ z grafu $p(x_1 ,x_2 , 1)$ sa rovná : 0.23301080280423597\newline
Táto hodnota je silne podobná s výsledkom z úlohy 11, nepresnosť mohla vzniknúť zaokrúhlovaním.
\end{document}